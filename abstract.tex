%%%%%%%%%%%%%%%%%%%%%%%%%%%%%%%%%%%%%%%%%
% Journal Article
% LaTeX Template
% Version 1.3 (9/9/13)
%
% This template has been downloaded from:
% http://www.LaTeXTemplates.com
%
% Original author:
% Frits Wenneker (http://www.howtotex.com)
%
% License:
% CC BY-NC-SA 3.0 (http://creativecommons.org/licenses/by-nc-sa/3.0/)
%
%%%%%%%%%%%%%%%%%%%%%%%%%%%%%%%%%%%%%%%%%

%----------------------------------------------------------------------------------------
%	PACKAGES AND OTHER DOCUMENT CONFIGURATIONS
%----------------------------------------------------------------------------------------

\documentclass[twoside]{article}

\usepackage{lipsum} % Package to generate dummy text throughout this template

%\usepackage[sc]{mathpazo} % Use the Palatino font
\usepackage{tgbonum}
\usepackage[T1]{fontenc} % Use 8-bit encoding that has 256 glyphs
\linespread{1.05} % Line spacing - Palatino needs more space between lines
\usepackage{microtype} % Slightly tweak font spacing for aesthetics

\usepackage[hmarginratio=1:1,top=32mm,columnsep=20pt]{geometry} % Document margins
\usepackage{multicol} % Used for the two-column layout of the document
\usepackage[hang, small,labelfont=bf,up,textfont=it,up]{caption} % Custom captions under/above floats in tables or figures
\usepackage{booktabs} % Horizontal rules in tables
\usepackage{float} % Required for tables and figures in the multi-column environment - they need to be placed in specific locations with the [H] (e.g. \begin{table}[H])
\usepackage{hyperref} % For hyperlinks in the PDF

\usepackage{lettrine} % The lettrine is the first enlarged letter at the beginning of the text
\usepackage{paralist} % Used for the compactitem environment which makes bullet points with less space between them

\usepackage{abstract} % Allows abstract customization
\renewcommand{\abstractnamefont}{\normalfont\bfseries} % Set the "Abstract" text to bold
%\renewcommand{\abstracttextfont}{\normalfont\small\itshape} % Set the abstract itself to small italic text

\usepackage{titlesec} % Allows customization of titles
\renewcommand\thesection{\Roman{section}} % Roman numerals for the sections
\renewcommand\thesubsection{\Roman{subsection}} % Roman numerals for subsections
\titleformat{\section}[block]{\large\scshape\centering}{\thesection.}{1em}{} % Change the look of the section titles
\titleformat{\subsection}[block]{\large}{\thesubsection.}{1em}{} % Change the look of the section titles

\usepackage{fancyhdr} % Headers and footers
\pagestyle{fancy} % All pages have headers and footers
\fancyhead{} % Blank out the default header
\fancyfoot{} % Blank out the default footer
\fancyhead[C]{CSC 4444 $\bullet$ Oct. $19^{th}$ $\bullet$ Abstract  $\bullet$ Group 3} % Custom header text
\fancyfoot[RO,LE]{\thepage} % Custom footer text

%----------------------------------------------------------------------------------------
%	TITLE SECTION
%----------------------------------------------------------------------------------------

\title{\vspace{-15mm}\fontsize{24pt}{14pt}\selectfont\textbf{Dream Catcher}} % Article title

\author{
\small
%\textsc
{Inna Limjoco, Jumao Yuan, Kyle Martinez,  Michael Chan, Miles Robicheaux and Shasha Jiang}%\thanks{A thank you or further information}\\[2mm] % Your name 
\\
%\vspace{10mm}
\normalsize Louisiana State University \\ 
% Your institution
%\normalsize \href{mailto:john@smith.com}{john@smith.com} % Your email address
\vspace{-5mm}
}
\date{}

%----------------------------------------------------------------------------------------

\begin{document}

\maketitle % Insert title

\thispagestyle{fancy} % All pages have headers and footers

%----------------------------------------------------------------------------------------
%	ABSTRACT
%----------------------------------------------------------------------------------------

\begin{abstract}
{\fontfamily{cmr}\selectfont
% font: https://www.sharelatex.com/learn/Font_typefaces

		Dreaming was considered a supernatural communication or a means of divine intervention, whose message could be unravelled by people with certain powers. Our project, Dream Catcher, will focus on dream interpretation and display it via Android Apps. Starting with 400 dream terms, such as accident, darkness, falling, flying, ocean, etc., we created interpretations for each term obtained from a dream dictionary. With IBM Watson system, we uploaded documents into corpus, which included both dream terms and its interpretations. With IBM Watson learning process, we generated 500 Q\&A pairs from the uploaded dream documents. For instance, if you ask IBM Watson "What does it mean when I dream about falling?", you can get several answers in different correlation levels. Generally, the top response is the most relevant answer to the question. \\

	In addition, the platform applied for users is Android Apps. Android provides an app framework (Android Studio) that allows developers to provide unique resources for different device configurations. For our application, we mainly use XML and JavaScript for Android programming.  The connection between IBM Watson and Android Apps is through rest API, which can be reached from IBM Bluemix. When users asked questions via an Android device, the question documents will be then delivered to the IBM Watson's system through the rest API. After IBM Watson finishes processing the question, it then transfers answers back to the users' Android device using the rest API once again. For example, when users ask "What does it mean when I dream about falling?",  a  JSON file will be delivered to the IBM Watson corpus. After the question is interpreted, it will deliver an answer to the users' Android device and show up on the interface. \\

	Our project is robust since dream interpretation is closely correlated to people's lives. Dreams are part of our subconscious. With this in mind, it can help you discover a part of you that you did not even know. With our Dream Catcher App, not only can you interpret your dreams, you can keep it for memories sake. Furthermore, you can share these mind-created stories to the world.  \\

}
%\noindent \lipsum[1] % Dummy abstract text
\noindent

\end{abstract}

%----------------------------------------------------------------------------------------
%	REFERENCE LIST
%----------------------------------------------------------------------------------------

\begin{thebibliography}{99} % Bibliography - this is intentionally simple in this template
%
%\bibitem[Figueredo and Wolf, 2009]{Figueredo:2009dg}
%Figueredo, A.~J. and Wolf, P. S.~A. (2009).
%\newblock Assortative pairing and life history strategy - a cross-cultural
%  study.
%\newblock {\em Human Nature}, 20:317--330.

\bibitem[1] [[Peirce Penney, 2008]
Peirce Penney, (2008).
\newblock Dream Dictionary For Dummies. 
\newblock {\em Wiley Publishing, Inc., Indianapolis, Indiana}.

%\bibitem[2] [[author, year]

\noindent


\end{thebibliography}
[3] \url{https://en.wikipedia.org/wiki/Dream_interpretation} \newline
[4] \url{http://www.ibm.com/smarterplanet/us/en/ibmwatson/} \newline
[5] \url{https://developer.android.com/guide/index.html} \newline

%----------------------------------------------------------------------------------------


\end{document}
