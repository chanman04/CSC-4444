%%%%%%%%%%%%%%%%%%%%%%%%%%%%%%%%%%%%%%%%%
% Journal Article
% LaTeX Template
% Version 1.3 (9/9/13)
%
% This template has been downloaded from:
% http://www.LaTeXTemplates.com
%
% Original author:
% Frits Wenneker (http://www.howtotex.com)
%
% License:
% CC BY-NC-SA 3.0 (http://creativecommons.org/licenses/by-nc-sa/3.0/)
%
%%%%%%%%%%%%%%%%%%%%%%%%%%%%%%%%%%%%%%%%%

%----------------------------------------------------------------------------------------
%	PACKAGES AND OTHER DOCUMENT CONFIGURATIONS
%----------------------------------------------------------------------------------------

\documentclass[twoside]{article}

\usepackage{lipsum} % Package to generate dummy text throughout this template

%\usepackage[sc]{mathpazo} % Use the Palatino font
\usepackage{tgbonum}
\usepackage[T1]{fontenc} % Use 8-bit encoding that has 256 glyphs
\linespread{1.05} % Line spacing - Palatino needs more space between lines
\usepackage{microtype} % Slightly tweak font spacing for aesthetics

\usepackage[hmarginratio=1:1,top=32mm,columnsep=20pt]{geometry} % Document margins
\usepackage{multicol} % Used for the two-column layout of the document
\usepackage[hang, small,labelfont=bf,up,textfont=it,up]{caption} % Custom captions under/above floats in tables or figures
\usepackage{booktabs} % Horizontal rules in tables
\usepackage{float} % Required for tables and figures in the multi-column environment - they need to be placed in specific locations with the [H] (e.g. \begin{table}[H])
\usepackage{hyperref} % For hyperlinks in the PDF

\usepackage{lettrine} % The lettrine is the first enlarged letter at the beginning of the text
\usepackage{paralist} % Used for the compactitem environment which makes bullet points with less space between them

\usepackage{abstract} % Allows abstract customization
\renewcommand{\abstractnamefont}{\normalfont\bfseries} % Set the "Abstract" text to bold
%\renewcommand{\abstracttextfont}{\normalfont\small\itshape} % Set the abstract itself to small italic text

\usepackage{titlesec} % Allows customization of titles
\renewcommand\thesection{\Roman{section}} % Roman numerals for the sections
\renewcommand\thesubsection{\Roman{subsection}} % Roman numerals for subsections
\titleformat{\section}[block]{\large\scshape\centering}{\thesection.}{1em}{} % Change the look of the section titles
\titleformat{\subsection}[block]{\large}{\thesubsection.}{1em}{} % Change the look of the section titles

\usepackage{fancyhdr} % Headers and footers
\pagestyle{fancy} % All pages have headers and footers
\fancyhead{} % Blank out the default header
\fancyfoot{} % Blank out the default footer
\fancyhead[C]{CSC 4444 $\bullet$ Oct. $19^{th}$ $\bullet$ Abstract  $\bullet$ Group 3} % Custom header text
\fancyfoot[RO,LE]{\thepage} % Custom footer text

%----------------------------------------------------------------------------------------
%	TITLE SECTION
%----------------------------------------------------------------------------------------

\title{\vspace{-15mm}\fontsize{24pt}{14pt}\selectfont\textbf{Dream Catcher}} % Article title

\author{
\small
%\textsc
{Inna Limjoco, Jumao Yuan, Kyle Martinez,  Michael Chan, Miles Robicheaux and Shasha Jiang}%\thanks{A thank you or further information}\\[2mm] % Your name 
\\
%\vspace{10mm}
\normalsize Louisiana State University \\ 
% Your institution
%\normalsize \href{mailto:john@smith.com}{john@smith.com} % Your email address
\vspace{-5mm}
}
\date{}

%----------------------------------------------------------------------------------------

\begin{document}

\maketitle % Insert title

\thispagestyle{fancy} % All pages have headers and footers

%----------------------------------------------------------------------------------------
%	ABSTRACT
%----------------------------------------------------------------------------------------

\begin{abstract}
{\fontfamily{cmr}\selectfont
% font: https://www.sharelatex.com/learn/Font_typefaces

		Dreaming was considered a supernatural communication or a means of divine intervention, whose message could be unravelled by people with certain powers. Our project, Dream Catcher, will focus on dream interpretation and display it via an Android mobile application. Starting with 400 dream terms, such as accident, darkness, falling, flying, ocean, etc., we created interpretations for each term obtained from a dream dictionary. We utilized IBM Watson system and uploaded documents into the document corpus, which included both dream terms and its interpretations. Meanwhile, we utilized the IBM Watson learning process and generated 500 Q\&A pairs from the uploaded dream documents. For instance, if you ask IBM Watson "What does it mean when I dream about falling?", you can get several answers in different correlation levels. Generally, the top response is the most relevant answer to the question. \\

	We built a native mobile app which utilizes the Android operating system and will support Android Lollipop and up. In order to build our app we used Google's development suite Android Studio to both develop and test our app. Our mobile application is written using the standard  Java and XML languages. In addition, our mobile app establishes a connection to IBM Watson through IBM's REST API and then sends user questions as JSON objects to IBM Bluemix and receives responses as JSON objects as well. For example, when users ask "What does it mean when I dream about falling?",  a  JSON object will be delivered to the IBM Watson corpus. After the question is interpreted, it will deliver an answer as a JSON object to the users' Android device and be displayed for them to read. The question and answer API allows our user's the ability to take their dream in all its complexity and analyze symbols and moments within their dream to provide them a better understanding of their subconscious. \\

	Our project is robust since dream interpretation is a complex analytical process. Our app allows people to discover a portion of themselves they did not even know. The Dream Catcher mobile application, not only allows users to interpret their dreams, but also log and record  their dreams in a journal format in order to have them or share them with others. This app allows a user to explore their own subconscious and psychology in order to have a more enriching life. \\
	
	}
%\noindent \lipsum[1] % Dummy abstract text
\noindent

\end{abstract}

%----------------------------------------------------------------------------------------
%	REFERENCE LIST
%----------------------------------------------------------------------------------------

\begin{thebibliography}{99} % Bibliography - this is intentionally simple in this template
%
%\bibitem[Figueredo and Wolf, 2009]{Figueredo:2009dg}
%Figueredo, A.~J. and Wolf, P. S.~A. (2009).
%\newblock Assortative pairing and life history strategy - a cross-cultural
%  study.
%\newblock {\em Human Nature}, 20:317--330.

\bibitem[1] [[Peirce Penney, 2008]
Peirce Penney, (2008).
\newblock Dream Dictionary For Dummies. 
\newblock {\em Wiley Publishing, Inc., Indianapolis, Indiana}.

%\bibitem[2] [[author, year]

\noindent


\end{thebibliography}
[2] \url{https://en.wikipedia.org/wiki/Dream_interpretation} \newline
[3] \url{http://www.ibm.com/smarterplanet/us/en/ibmwatson/} \newline
[4] \url{https://developer.android.com/guide/index.html} \newline

%----------------------------------------------------------------------------------------


\end{document}
